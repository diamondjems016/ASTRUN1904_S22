\documentclass[11pt]{article}
\usepackage[includeheadfoot, top=1.0in, bottom=1.0in, hmargin=1.0in]{geometry}
\usepackage[utf8]{inputenc}
\usepackage{fancyhdr}
\usepackage{url}
\pagestyle{fancy}
\usepackage{setspace}
\usepackage{tabularx}


\lhead{Astronomy Lab II}
\rhead{Spring 2022}
\lfoot{Mead}
\rfoot{Mon 6-9pm}
\cfoot{\thepage}

\begin{document}
%%%%%%%%%%%%%%%%%%%%%%% INTRO %%%%%%%%%%%%%%%%%%%%%%%
\begin{center}
\LARGE{ASTR1904: Lab II -- Section 001} \\ \medskip \Large{Syllabus}\\ 
\end{center}

\noindent
\textbf{Instructor:} Jennifer Mead (jennifer.mead@columbia.edu)\\
\textbf{Office:} Pupin 1333 (office hours by appointment)\\ 
\textbf{Time:} {Mondays, 6:00-9:00 PM} \\
\textbf{Location:} {Astronomy Library, Pupin 1402} \\

%%%%%%%%%%%%%%%%%%%%%%% OVERVIEW %%%%%%%%%%%%%%%%%%%%%%%
\section*{Class Overview}
Welcome to Astronomy Lab II! This class corresponds to \textit{Stars and Atoms}, \textit{Theories of the Universe: Babylon to the Big Bang}, and \textit{Stars, Galaxies, and Cosmology}.  The objectives of this lab are for you to:
\begin{itemize}
\item Develop critical thinking skills and learn to apply scientific reasoning in your evaluation of information and arguments. 
\item Develop a better sense of the scale of the Universe, an understanding of error and uncertainty in measurements, and other quantitative tools.
\item Learn to frame and ask well-defined questions, gather data and test that question quantitatively, model astronomical phenomena, and communicate your findings in both written and verbal formats.
\end{itemize}

\bigskip
 
\noindent There will be 10 labs throughout the semester.  There will be no lab work assigned outside of class. The final class will be devoted to student presentations on a topic of your choice.
 
%%%%%%%%%%%%%%%%%%%%%%% MATERIALS %%%%%%%%%%%%%%%%%%%%%%%
\section*{Lab Materials}
 
Please bring the following to each lab session:
 
\begin{itemize}
\item \textbf{A lab notebook:} This can be a bound notebook, but feel free to use a electronic document instead; if you choose the latter, make sure to have your device with you and ready to use.  
\item \textbf{Writing/drawing tools:} Pen, pencil, eraser, ruler, etc. Colored pens/pencils may come in handy but are not required.
\item \textbf{Scientific calculator:}  A calculator capable of performing trigonometric functions, logarithms, exponents, roots, etc. A graphing calculator is not required. 
\item \textbf{Laptop:} Laptops will be a necessity for many of the labs.  A limited number of laptops will be available for students who don't have their own. \\
\end{itemize}

%%%%%%%%%%%%%%%%%%%%%%% GRADING %%%%%%%%%%%%%%%%%%%%%%%
\section*{Grading}

\subsection*{Lab Write-ups}
Each lab will clearly denote what you should record in your write-ups for each lab. Lab responses can be recorded in either a bound physical notebook or in an electronic document. You may submit your work either as a \textbf{PDF} to Courseworks (strongly preferred), or hand in your lab notebook to the instructor at the end of class, to be returned at beginning of the following lab.  All submissions will be due by midnight on the day of the lab.

\noindent While I strongly recommend you work with a partner, each of you should keep your own records. The entire goal of the write-ups is to explain to the instructor \textit{what} you did during the lab, \textit{how} you did it, and \textit{why} you did it --- we are much more concerned with your reasoning behind your arguments than we are about the format.

\subsection*{Participation}
\noindent Participation is an essential part of this lab. You will often work with your peers, though students will be given the option to work alone if they prefer to do so given COVID-19. Your participation grade will be based on your contribution to your group, if you come to lab prepared and on time, and the questions your ask and/or attempt to answer.

\subsection*{Final Presentations}
\noindent For the final session, each student will give a 10-minute presentation on a topic or their choice followed by a 5-minute discussion with the class. A list of topics related to astronomy and science in society will be provided, but you are also welcome to submit your own suggested topics, pending my approval.

\subsection*{Grade Breakdown}
\noindent \textbf{75\%} Lab submissions*

\noindent \textbf{15\%} Final Presentations

\noindent \textbf{10\%} Participation

\noindent *Your lowest lab grade will be dropped when determining your final grade.

%%%%%%%%%%%%%%%%%%%%%%% SCHEDULE %%%%%%%%%%%%%%%%%%%%%%%
\section*{Tentative Schedule}

\begin{tabular}{cc}
    1/24 & (On Zoom) Lab 1: Orders of Magnitude\\
    1/31 & Lab 2: The Multiwavelength Universe\\
    2/07 & Lab 3: Spectroscopy\\
    %chemical composition, surface temp, spectral classification, stellar motion, energy levels of electrons\\
    2/14 & Lab 4: H-R Diagrams\\
    2/21 & Lab 5: Galaxy Classification\\
    2/28 & Lab 6: Dark Matter\\
    3/07 & No Lab / Make-up Labs\\
    3/14 & Happy Spring Break!\\
    3/21 & Lab 7: Exoplanet Transits\\
    3/28 & Lab 8: Hubble's Law and the Distance Ladder\\
    4/04 & Lab 9: Supernovae\\
    4/11 & Lab 10: Numerical Methods\\
    4/18 & Presentation Prep / Make-up Labs\\
    4/25 & Presentations\\
    5/02 & No Class - Good Luck on Finals!!
\end{tabular}

%%%%%%%%%%%%%%%%%%%%%%% POLICIES %%%%%%%%%%%%%%%%%%%%%%%
\section*{Policies}
 
\subsection*{Attendance}
 
By department policy, more than two unexcused (non-medical related) absences will result in automatic failure of the course. Please notify me if  extenuating circumstances arise (family emergencies, serious illness, quarantine requirement, or religious holidays) and we will arrange a make-up lab.
 
\subsection*{Accommodations}
If you have an identified disability, we encourage you to register with the Office of Disability Services: https://health.columbia.edu/services/register-disability-services
 
\subsection*{Academic Honesty}
https://www.college.columbia.edu/academics/academicintegrity
 
\subsection*{Mandatory reporting}
Instructors are required to report allegations of ``gender based misconduct, discrimination, or harassment" to Columbia's administration. While we are willing to listen and seek out resources (including confidential counselors) on your behalf, we cannot ourselves provide confidentiality.

\section*{Astronomy events at Columbia:}
Public lectures and observing sessions: http://outreach.astro.columbia.edu/

\end{document}
