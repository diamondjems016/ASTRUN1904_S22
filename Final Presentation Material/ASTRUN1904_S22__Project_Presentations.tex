\documentclass[11pt]{article}
\usepackage[includeheadfoot, top=1.0in, bottom=1.0in, hmargin=1.0in]{geometry}
\usepackage[utf8]{inputenc}
\usepackage{fancyhdr}
\usepackage{url}
\pagestyle{fancy}
\usepackage{setspace}
\usepackage{tabularx}
\usepackage{graphicx}
\usepackage{caption}
\usepackage{subcaption}
\usepackage{hyperref}
\usepackage{multicol}
\usepackage{amsmath}
\usepackage{enumitem}

\usepackage{hyperref}
\hypersetup{
    colorlinks=true,
    linkcolor=blue,
    filecolor=magenta,      
    urlcolor=blue,
}


\lhead{Astronomy Lab II}
\rhead{Spring 2022}
\lfoot{Mead}
\rfoot{Mon 6-9pm}
\cfoot{\thepage}

\begin{document}

\begin{center}
\huge{Project Presentations}\\ \medskip \Large{April 25}
\end{center}

\section{Overview}
For our last lab (\textbf{April 25}), each of you will give a presentation of an astronomy topic of your choice.  I have provided a list of suggested topics here, but you are welcome to come up with a topic of your own.  All topics must be approved by me by \textbf{April 11}, and no two topics may be the same, so they will be approved on a first-come, first-serve basis.  This is your opportunity to explore something that intrigues \textit{you} about astronomy and share that with me and your classmates, so have fun with your presentations!  Class on April 18th is dedicated to preparation. I'm also available by email or appointment if you would like to get feedback or do a dry-run of your talk.

\section{Guidelines}
\subsection*{Preparation}
\begin{itemize}
    \item You should submit the list of research papers, popular articles, websites, books, chapters you referred to by \textbf{5 PM on April 24}, along with your slides. You should include references in your presentation that you used at appropriate locations. No special formatting or citation style is needed
    \item Keep in mind that it is often more compelling to discuss a few sub-topics that you know well, versus putting lots of content that is glossed over/simply read off your slides
\end{itemize}
\subsection*{Presentations}
\begin{itemize}
    \item Presentations should be \textbf{10-15 minutes in length}
    \item Each presentation will be followed by a \textbf{5 minute question period}
    \item You may use any combination of slides and/or whiteboard
    \item Everyone should answer the following questions for each talk; your feedback will be given to the presenter. Printed forms will be provided on the day of the presentation.
    \begin{itemize}
        \item What's one thing you learned and/or enjoyed?
        \item What's one strength of the presentation that aided clarity, engagement?
        \item If you were to give the same talk, what would you change to convey
            the ideas more clearly?
    \end{itemize}
    \item Come ready to ask questions during and after each talk; these will count for participation.  Any kind: E.g., ``I didn't understand your sentence just now, as it is contradictory with previous statement'', asking for more information, hypothetical questions based on relevant scenarios, etc.~will count. More thought-out the questions are better. But, we believe that no question is a bad question.
\end{itemize}

\subsection*{Grading}
15\% of your final grade -- 10\% for the presentations, and 5\% for your participation. Here is a rubric\footnote{Chiefly adapted from the American Astronomical Society---Chambliss award rubric.} for your presentation.

\noindent
\textbf{Content: 70\%}
\begin{itemize}
%\itemsep0em
\item (35\%) Presenter introduces and describe(s) topic at level appropriate to this class [\underline{\hspace{5mm}}]
\item (40\%) Presenter explains extent of and limitations on our knowledge on the topic, including data/observations underlying knowledge [\underline{\hspace{5mm}}]
\item (20\%) Presenter provides context by drawing connections to, e.g., different areas of astronomy, concepts from lab or
lecture, other areas of science, areas outside of science, etc. [\underline{\hspace{5mm}}]
\item (5\%) Presenter chooses and cites appropriate references (i.e., goes beyond Wikipedia and popular press releases).  Presenter submits reference list. [\underline{\hspace{5mm}}]
\end{itemize} 

\noindent
\textbf{Delivery: 30\%}
\begin{itemize}
\item (35\%) Presentation has a logical flow that audience can follow [\underline{\hspace{5mm}}]
\item (25\%) Presenter can address reasonable audience questions [\underline{\hspace{5mm}}]
\item (20\%) Presentation aids (slides or board-work) are understood by audience [\underline{\hspace{5mm}}]
\item (10\%) Presenter stays within allotted time [\underline{\hspace{5mm}}]
\item (10\%) Presenter speaks clearly, and keeps the audience engaged (questions, activities, etc.) [\underline{\hspace{5mm}}]
\end{itemize}
{\small [\underline{\hspace{5mm}}] = easily and concisely (4), sufficiently (3),
is somewhat able to (2), barely to did not (1)
}
\pagebreak
\section{Suggested topics}

Please submit your proposed topics by \textbf{10 PM on April 11th}.

\medskip \noindent
A non-comprehensive list of suggested topics follows.  You can choose something not listed, so long as it's within the realm of stars, galaxies, cosmology, and related topics relevant to our lab's focus area. It should be something you haven't covered in depth in class or this lab.

\medskip \noindent
I recommend you go one step deeper for most of the below suggestions.

\medskip \noindent
\textbf{Good topic:} ``The Great Red Spot and other storms, vortices, and zonal flows on
Jupiter''.

\noindent
\textbf{Not-as-good topic:} ``Gas giant atmospheres''.
This will help both you and me determine whether your topic is well-suited for
a 10--15 min presentation.

\begin{itemize}[noitemsep]
    \item Galaxies (including our own)
        \begin{itemize}[noitemsep]
            \item Supermassive black hole and galactic dynamics (birth, growth, rotation, etc.)
            \item Different theories of dark matter
            \item Intergalactic medium
            \item Galactic halo, and dark matter content of different galaxies
            %\item Stellar life cycle---from birth to supernova!
            \item Dark energy
        \end{itemize}

    \item Stars (including Sun)
        \begin{itemize}[noitemsep]
            \item Interior structure, chemistry and phase composition
            \item Asteroseismology (aka starquakes!)
            \item Surface properties, and stellar atmospheres, and magnetospheres
            \item The process of star formation, and star forming regions in galaxies
            %\item Stellar life cycle---from birth to supernova!
            \item Binary star systems
            \item Specific types of star (Wolf-Rayet, T-Tauri, RR Lyrae, Population III aka the first stars)
        \end{itemize}

    \item Planets
        \begin{itemize}[noitemsep]
            \item Solar system formation and history
            \item Proto-planetary disks
            \item Planet and planetesimal formation
            \item Brown dwarves
            \item Exoplanets: Types, detection methods, atmosphere, future missions, etc.
            \item Are we alone?
        \end{itemize}
    
    \item Telescopes and spacecraft
        \begin{itemize}[noitemsep]
            \item Ground- versus space-based telescopes
            \item Specific missions/projects: Hubble, James Webb Space Telescope, Kepler, Very-long-baseline interferometry, Rossi X-ray Timing Explorer (RXTE), Large Synoptic Survey Telescope, Chandra (or your other favorite choice).
            \item NASA budget, mission, proposals.  How funding decisions are made.
        \end{itemize}

    \item Spacefaring; Search for Extraterrestrial Intelligence (SETI)
    \begin{itemize}[noitemsep]
        \item Astrobiology, chemistry; the habitable zone
        \item Speciation and extinctions on Earth
        \item Energy usage, Dyson spheres
        \item Communication and signal detection; candidate SETI signals
        \item Space travel; Breakthrough Starshot
    \end{itemize}

    \item Miscellaneous
        \begin{itemize}[noitemsep]
            \item Gravitational waves and LIGO
            \item Compact objects (Black hole, neutron stars, white dwarf)
            \item Mysterious signals (Fast Radio Bursts, Gamma Ray Bursts)
            \item Big bang, and the acceleration of Universe (inflation, nucleosynthesis, final fate, etc.)
            \item Clusters (of stars and galaxies)
            \item Present any scientific paper. I recommend looking at \url{https://arxiv.org/archive/astro-ph} to choose your favorite field within astronomy. Then go through some of the papers' titles to narrow down to a paper of your liking.
            \item Biographical study of a famous astronomer or planetary scientist. If you do this, choose at least one scientific contribution to emphasize.
            \begin{itemize}[noitemsep]
                \item Galileo, Kepler, etc.
                \item Caroline Herschel (comets)
                \item Annie Maunder (sunspots, solar corona, eclipses)
                \item Annie Jump Cannon (spectra of stars)
                \item Cecilia Payne-Gaposchkin (the composition of stars)
                \item Vera Rubin (dark matter)
                \item David Jewitt (trans-Neptunian objects, comets)
                \item Margaret Kivelson (solar wind, Europa’s ocean)
                \item Carl Sagan (science communication; solar system, astrobiology)
                \item Jill Tarter (SETI)
                \item Sara Seager (exoplanets)
            \end{itemize}
            
        \end{itemize}
        
\end{itemize}

%Other options (equally encouraged):
%\begin{itemize}    
    
    %I recommend looking at the Daily Paper
%        Summaries on Astrobites (\url{https://astrobites.org/}).  This is a
%        blog that summarizes scientific papers at an introductory level;
%        summaries are written by astro graduate students and aimed for
%        undergrad/grad students alike.
%        Other sources of brief, accessible scientific papers include Nature
%        (\url{https://www.nature.com/}), Nature Astronomy
%        (\url{https://www.nature.com/natastron/}), and
%        Science (\url{https://www.sciencemag.org/}).
%        For popular press that can direct you to interesting papers, consider:
%        \url{https://www.quantamagazine.org/physics} or
%        \url{https://www.scientificamerican.com}.

%    \item Biographical study of a famous astronomer or planetary scientist.
%        If you do this, choose at least one scientific contribution to
%        emphasize.
%        \begin{itemize}[noitemsep]
%            \item Galileo, Kepler, etc.
%            \item Caroline Herschel (comets)
%            \item Annie Maunder (sunspots, solar corona, eclipses)
%            \item Annie Jump Cannon (spectra of stars)
%            \item Cecilia Payne-Gaposchkin (the composition of stars)
%            \item David Jewitt (trans-Neptunian objects, comets)
%            \item Margaret Kivelson (solar wind, Europa's ocean)
%            \item Carl Sagan (science communication; solar system,
%                astrobiology)
%            \item Jill Tarter (SETI)
%            \item Sara Seager (exoplanets)
%        \end{itemize}
%\end{itemize}

\end{document}



\section{Suggested Topics}

\end{document}